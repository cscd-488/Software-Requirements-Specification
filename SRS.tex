\documentclass{scrreprt}
\usepackage{listings}
\usepackage{underscore}
\usepackage[bookmarks=true]{hyperref}
\usepackage[utf8]{inputenc}
\usepackage[english]{babel}
\hypersetup{
    bookmarks=false,                                % show bookmarks bar?
    pdftitle={Software Requirement Specification},  % title
    pdfauthor={Joshua D. Harshman},                 % author
    pdfsubject={TeX and LaTeX},                     % subject of the document
    pdfkeywords={TeX, LaTeX, graphics, images},     % list of keywords
    colorlinks=true,                                % false: boxed links; true: colored links
    linkcolor=blue,                                 % color of internal links
    citecolor=black,                                % color of links to bibliography
    filecolor=black,                                % color of file links
    urlcolor=purple,                                % color of external links
    linktoc=page                                    % only page is linked
}%
\def\myversion{2.2 }
\date{}
%\title
\usepackage{hyperref}
\begin{document}

\begin{flushright}
    \rule{16cm}{5pt}\vskip1cm
    \begin{bfseries}
        \Huge{SOFTWARE REQUIREMENTS\\ SPECIFICATION}\\
        \vspace{1.9cm}
        for\\
        \vspace{1.9cm}
        Magpie\\
        \vspace{1.9cm}
        \LARGE{Version \myversion}\\
        \vspace{1.9cm}
        Prepared by Joshua D. Harshman\\Bruce Emehiser\\Aaron Young\\Andrew Macy\\
        \vspace{1.9cm}
        Application Development Team\\
        \vspace{1.9cm}
        \today\\
    \end{bfseries}
\end{flushright}

\tableofcontents

\chapter*{Revision History}

\begin{center}
    \begin{tabular}{|c|c|c|c|}
        \hline
	    Name & Date & Reason For Changes & Version \\
        \hline
	    Rough Draft & 2016 02 08 & Initial SRS & 1.0 \\
        \hline
        Version 2.0 update & 2016 02 21 & Prep for client release & 2.0 \\
        \hline
        Version 2.2 update & 2016 06 05 & Modification for Alpha release & 2.2 \\
        \hline
    \end{tabular}
\end{center}

\chapter{Introduction}

\section{Purpose}
The purpose of this document is to give a detailed description of requirements for
the Magpie mobile application and companion web services. It will illustrate the purpose
and complete declaration for the development of the system.  It will also explain system
constraints, interface, interactions, and limitations.

\section{Document Conventions}
Not applicable at this time \\ \\
See 6.0.1 and 6.0.2 for for Coding Conventions

\section{Intended Audience and Reading Suggestions}
This document is intended for all those interested or having stake in the Magpie
application.  Companies planning to use this product as well as future developers
planning to contribute or elaborate on this software will find the following documentation
helpful to understand the specifications and functions of both the mobile application and server application.

\section{Project Scope}
The Magpie mobile application allows users to track their visitation to events,
receive digital badges and exchange said badges for real life rewards. Badges are awarded
to users based on proximity to available events in their area.  Checking in via GPS or
if applicable by scanning a QR code, will yield a digital badge that is then stored for them in a resilient database on Magpie's servers.

\section{References}
Not applicable at this time

\chapter{Overall Description}

\section{Product Perspective}
The Magpie app was first envisioned as a standalone mobile application that one
could use to track their progress through a given event and obtain badges for doing so.
It has since grown into a much bigger and more ambitious, multi-part project. And while the
fundamental idea remains the same, the implementation is turning into a robust, social, and
educational piece of software that has it's roots in game theory.

\section{Product Functions}
Magpie is intended to allow users to find interesting locations, to influence them
to visit those locations, to share the experience with friends, and collect badges
which they can redeem for some nominal reward. Because of this, we have the
following functions:
\begin{itemize}
\item User authentication
\item Robust web service backend
\item GPS and QR location check in
\item Social media hooks for sharing events and waypoints
\end{itemize}
With these features, users will be allowed to log in to the app, view events,
view waypoints, share events and waypoints via social media, and collect
badges at waypoints via GPS and QR.

\section{User Classes and Characteristics}
Magpie really has something for everyone. The game-ification of collecting badges to
exchange for real life rewards will drive the younger generations to use the application.
Not to mention the social media integration that will allow the user to quickly share, tweet, and
post their latest adventure.

\section{Operating Environment}
Magpie will run on Android API 16 and newer android devices. It is designed for
mobile phones, but it can also be run on Android tablets, although the screen
sizes are not specifically supported. The phones/tablets must have a rear facing
camera to be able to utilize the QR code check in feature.

\section{Design and Implementation Constraints}
A minimum Android version of 4.3 (Jelly Bean) is required to access *all* features of the Magpie application. It is up to the end user to develop and maintain a database and server for the application to interface with as well as maintain a way to push events to the Magpie server.

\section{Assumptions and Dependencies}
Magpie is dependent on a database back end for pulling events and updating
content. Without the back end it is incapable of pulling event data and
providing the user with information.


\chapter{External Interface Requirements}

\section{User Interfaces}

The user interface will be built with Android XML layouts. It will feature
\begin{itemize}
\item Paging introduction view
\item Share image drawer
\item Login screen
\item Event view screen
\item Way-point view screen
\item QR code scanner screen
\end{itemize}

\section{Hardware Interfaces}
Each device must have a rear-facing camera capable of auto-focus,
and and producing JPEG photos. This will be used by the QR code scanner for event waypoint checkin.
Magpie will also need network connection to send and receive user data to it's server.
The device must also be GPS capable if the user wants to take advantage of the GPS check in and event location features.

\section{Software Interfaces}
Magpie will make use of the following software interfaces
\begin{itemize}
  \item Mobile Application
  \begin{itemize}
    \item ZXing QR code API
    \item Google Plus API
    \item Google Play Services
    \item hdodenhof's circleimageview library
    \item koushikdutta's ION library
  \end{itemize}
  \item Server Application
  \begin{itemize}
    \item FreeBSD 10.2
    \item NGINX
    \item MySQL
    \item PHP 5.7
  \end{itemize}
\end{itemize}

\section{Communications Interfaces}
Magpie will communicate with the server via HTTPS connections, for getting event
data, and storing user data. \\
Magpie will also communicate with social media outlets Google (for user authentication),
and Google+, Facebook, Twitter, and Instagram for social media sharing via the
APIs provided.

\chapter{System Features}
These are the main features included in Magpie.

\section{System Feature 1 : Intro \& Login}
Login screen with paginated intro / tutorial

\subsection{Description and Priority}
Provides a login as well as a quick tutorial outlining the app's  basic features and
usage. This is of mixed priority, the login acts as the initial entry point for new users
which is of high priority, while the paginated tutorial provides a nice feature to user
experience which is not critical to the application's overall functionality.

\subsection{Stimulus/Response Sequences}
On initial launch, Magpie will present the user with a professionally designed login
page providing the user the option for logging in with their Google account. The user
will also be able to swipe the screen to view a paginated tutorial highlighting the
usage and main features of the application.

\subsection{Functional Requirements}
Server side database backend assisting with login verification.

\section{System Feature 2 : Events}

\subsection{Description and Priority}
Here begins the core functionality of the application and is as such high priority.

\subsection{Stimulus/Response Sequences}
Continuing the first run sequence, the application will first prompt the user for their location.
The user can either enter a location manually or touch the GPS button to enable auto-detection.
This location will be saved as a setting and stored in app data.
An asynchronous task will run in the background, updating the device's local list of events and waypoints.
While this asynchronous task runs, the Magpie loading screen will display to the user.

\subsection{Functional Requirements}
Server side database back end for pulling event data.

\section{System Feature 3 : Checkin}

\subsection{Description and Priority}
User checkin is a core feature of the app. Users will be allowed to check in to event way points by using GPS location and QR code scans.

\subsection{Stimulus/Response Sequences}
User will select an event and be presented with a list of event way points. User will then
select a single way point which they wish to check in to. They will press a button and either
be checked in via GPS or a Camera will be shown which they can use to scan a QR code. \\
If GPS is disabled, user will be prompted to allow access. \\
If Camera is busy, user will be notified.

\subsection{Functional Requirements}
GPS connectivity - For full functionality\\
Internet connection \\
Camera and ZXing API for QR code scanning.

\section{System Feature 4 : Badges}
One key feature of the app is the collection of badges. These badges, when collected,
can be used to redeem some item of nominal value from a vendor/booth, such as a
cup of coffee, for visiting all way points. The app will allow for the viewing of collected
badges, as well as a way of signifying when a badge or set of badges has been redeemed.

\subsection{Stimulus/Response Sequences}
User will present phone to vendor. Vendor will be able to scan screen, or otherwise
see and redeem collected badges.

\subsection{Functional Requirements}
Database connection for loading collected badges from. Possible vendor database connection
for claiming badges from a user.


\chapter{Other Nonfunctional Requirements}

\section{Performance Requirements}
Internet connection must be suitable for loading images and other content from a
database. A simple wifi or 2G connection should be sufficient. \\
Phone must be powerful enough to decode QR codes from images.

\section{Safety Requirements}
Setting up user data handling is still in research.

\section{Security Requirements}
Network security is still in research.

\section{Software Quality Attributes}
The software is built to be extensible. It can be extended to almost any event at
which there are booths, attractions, or other locations or points of interest. \\
Administrators will be able to easily add events to the server, and users will see
and be able to select and pull new events. \\
Magpie uses Android Fragments to make the code modular and easily extensible.
This allows new modules to easily be added for increased functionality. \\
Maintenance should be small, with the only necessary maintenance being in
database maintenance, and security updates. \\

\chapter{Other Requirements}


\subsection{Naming Conventions}
\begin{itemize}
\item All code is to follow standard Java code naming conventions.
\item Variable names will be in Camel Case, beginning with a lower case letter.
\texttt{int variableName;}
\item Method names and Class names are to be camel case, beginning with an upper case letter. \\
\texttt{public class ClassName \{\}}
\item Member level variables are to be in Camel Case, and begin with a lower case character 'm'. \\
\texttt{private int mMemberLevelVariable;}
\end{itemize}

\subsection{Documentation Conventions}
All files are to contain sufficient documentation to be clearly understandable. This
contains but is not limited to
\begin{itemize}
\item File comments. Each file should contain a comment at the top which contains,
\item File name
\item Author name(s)
\item Date created
\item Date(s) modified
\item brief description of the file contents
\end{itemize}

\texttt{/** \\}
\texttt{* @file FileName.java \\}
\texttt{* @author FirstName LastName\\}
\texttt{* @date 2016 01 15 \\}
\texttt{* @date 2016 02 08\\}
\texttt{* This short description is intended to fulfill the requirements of an SRS. \\}
\texttt{*/ \\}

\begin{itemize}
\item Class Comments. All classes should contain a multi-line comment immediately above them which adheres to
the javadoc standards. They should contain a description of the Class and its intended function. \\
\end{itemize}

\texttt{/** \\}
\texttt{* Class Description used to show the proper documentation \\}
\texttt{* of a class \\}
\texttt{*/}
\texttt{public class ClassName \{\}}

\begin{itemize}
\item Method Comments. All methods should have a multi-line comment immediately with
\item Description
\item Parameter(s)
\item Return
\item Exceptions
\end{itemize}

\texttt{/** \\}
\texttt{* Method description describing the basic features \\}
\texttt{* and function of method\\}
\texttt{* \\}
\texttt{* @param paramName This parameter is useful because it shows param notation in comments \\}
\texttt{* @param param2 This shows the second parameter \\}
\texttt{* \\}
\texttt{* @return This is the description of the return value\\}
\texttt{* \\}
\texttt{*/}
\texttt{public class ClassName \{\}}

\end{document}
