\documentclass{scrreprt}
\usepackage{listings}
\usepackage{underscore}
\usepackage[bookmarks=true]{hyperref}
\usepackage[utf8]{inputenc}
\usepackage[english]{babel}
\hypersetup{
    bookmarks=false,                                % show bookmarks bar?
    pdftitle={Software Requirement Specification},  % title
    pdfauthor={Joshua D. Harshman},                 % author
    pdfsubject={TeX and LaTeX},                     % subject of the document
    pdfkeywords={TeX, LaTeX, graphics, images},     % list of keywords
    colorlinks=true,                                % false: boxed links; true: colored links
    linkcolor=blue,                                 % color of internal links
    citecolor=black,                                % color of links to bibliography
    filecolor=black,                                % color of file links
    urlcolor=purple,                                % color of external links
    linktoc=page                                    % only page is linked
}%
\def\myversion{0.1 }
\date{}
%\title
\usepackage{hyperref}
\begin{document}

\begin{flushright}
    \rule{16cm}{5pt}\vskip1cm
    \begin{bfseries}
        \Huge{SOFTWARE REQUIREMENTS\\ SPECIFICATION}\\
        \vspace{1.9cm}
        for\\
        \vspace{1.9cm}
        Event!\\
        \vspace{1.9cm}
        \LARGE{Version \myversion approved}\\
        \vspace{1.9cm}
        Prepared by Joshua D. Harshman\\Bruce Emehiser\\Aaron Young\\Andrew Macy\\
        \vspace{1.9cm}
        Application Development Team\\
        \vspace{1.9cm}
        \today\\
    \end{bfseries}
\end{flushright}

\tableofcontents


\chapter*{Revision History}

\begin{center}
    \begin{tabular}{|c|c|c|c|}
        \hline
	    Name & Date & Reason For Changes & Version\\
        \hline
	    Rough Draught & 2016 02 08 & Initial SRS & .01\\
        \hline
    \end{tabular}
\end{center}

\chapter{Introduction}

\section{Purpose}
The purpose of this document is to give a detailed description of requirements for
the "Event!" mobile application and companion web services. It will illustrate the purpose
and complete declaration for the development of the system.  It will also explain system
constraints, interface, interactions, and limitations. 

\section{Document Conventions}
Not applicable at this time

\section{Intended Audience and Reading Suggestions}
The intdended audience of this document are all those who have a a stakehold in
the "Event!" application and the developers.

\section{Project Scope}
The "Event!" mobile application allows users to track their visitation to events,
receive digital badges and exchange said badges for real life rewards.

\section{References}
Not applicable at this time

\chapter{Overall Description}

\section{Product Perspective}
The Event app was first envisioned as a standalone mobile application that one
could use to track their progress through a given event and obtain badges for doing so.
It has since grown into a much bigger and more ambitious project ...

\section{Product Functions}
Event! is intended to allow users to find interesting locations, to influence them 
to visit those locations, to share the experience with friends, and collect badges 
which they can redeem for some nominal reward. Because of this, we have the 
following functions: 
\begin{itemize}
\item User authentication
\item GPS and QR location check in
\item Social media hooks for sharing events/waypoints
\end{itemize}
With these features, users will be allowed to log in to the app, view events, 
view waypoints, share events and waypoints via social media, and collect 
badges at waypoints via GPS and QR.

\section{User Classes and Characteristics}
Event! appeals to all age ranges, with emphasis on the young who will be visiting 
events, art walks, and the like.

\section{Operating Environment}
Event! will run on Android API 16 and newer android devices. It is designed for 
mobile phones, but it can also be run on Android tablets, although the screen 
sizes are not spesifically supported. The phones/tablets must have a rear facing 
camera to be able to utilize the QR code check in feature.

\section{Assumptions and Dependencies}
The app is dependent on a database back end for pulling events and updating 
content. Without the back end it will be incapable of pulling event data and 
providing the user with information.


\chapter{External Interface Requirements}

\section{User Interfaces}

The user interface will be built with Android XML layouts. It will feature
\begin{itemize}
\item Paging introduction view
\item Navigation drawer
\item Login screen
\item Event view screen
\item Waypoint view screen
\item QR code scanner screen
\end{itemize}

\section{Hardware Interfaces}
Each device must have a rear-facing camera capable of producing JPEG photos, 
and auto focus. This will be used by the QR code scanner for event waypoint checkin.
Event! will also use network to send and recieve information from a server.

\section{Software Interfaces}
Event! will use ZXing QR code scanner API for decoding and creating QR codes for 
event and waypoint checkin. \\
Event! will use Apache server with PHP and MySQL database for maintaining event 
information and user information.

\section{Communications Interfaces}
Event! will communicate with the server via HTTPS connections, for getting event 
data, and storing user data. \\
The app will also communicate with social media outlets Google (for user authentication), 
and Google+, Facebook, Twitter, and Instagram for social media sharing via the 
APIs provided by the companies.

\chapter{System Features}
These are the main features included in Event!.

\section{System Feature 1 : Intro \& Login}
Paging intro screen

\subsection{Description and Priority}
Provides a quick tutorial to anyone installing the app on the basic features and 
usage of the app. This is of low priority, because it is not necessary for the functioning 
of the app, only the user experience.

\subsection{Stimulus/Response Sequences}
User will install app and be greeted by a quick tutorial. They can then choose to 
continue through the tutorial, or skip the tutorial. \\
After viewing the tutorial the user will be prompted to login with their Google account.
They will be able to click a single button and authorize the app.

\subsection{Functional Requirements}
Database back end for storing user login information. Users must posess a Google 
account, and should have one because it is required for an Android phone.

\section{System Feature 2 : Events}

\subsection{Description and Priority}
This contains the core functionality of the app, and has a very high priority. Upon opening 
the app, it will check for updates to the current events, and the user will be able to 
select new events.

\subsection{Stimulus/Response Sequences}
User will open app. Asyncronist task will run in the background, updating the device's
local database with events.

\subsection{Functional Requirements}
Database back end for pulling event data.

\section{System Feature 3 : Check-In}

\subsection{Description and Priority}
User checkin is a core feature of the app. Users will be allowed to check in to event waypoints
by using GPS location and QR code scans.

\subsection{Stimulus/Response Sequences}
User will select an event and be presented with a list of event waypoints. User will then
select a single waypoint which they wish to check in to. They will press a button and either
be checked in via GPS, or a Camera will be shown which they can use to scan a QR code. \\
If GPS is disabled, user will be prompted to allow access. \\
If Camera is busy, user will be notified.

\subsection{Functional Requirements}
GPS connectivity \\
Camera hardware and ZXing API for QR code scanning.

\section{System Feature 4 : Redeeming Badges}
One key feature of the app is redeeming the collected badges for some item of nominal value from
a vendor/booth, such as a cup of coffee, for visiting all waypoints at an event. The app will allow
for the viewing of collected badges, as well as a way of signifying when a badge or set of badges
has been redeemed.

\subsection{Stimulus/Response Sequences}
User will present phone to vendor. Vendor will be able to scan screen, or otherwise 
see and redeem collected badges.

\subsection{Functional Requirements}
Database connection for loading collected badges from. Possible vendor database connection 
for claiming badges from a user.


\chapter{Other Nonfunctional Requirements}

\section{Performance Requirements}
Internet connection must be suitable for loading images and other content from a 
database. A simple wifi or 2G connection should be sufficient. \\
Phone must be powerful enough to decode QR codes from images.

\section{Safety Requirements}


\section{Security Requirements}
Users data will be transfered via HTTPS secure connections. Users will be logged in
with a unique secure token created by Google's authentication services, which will
be stored on the apps private server space. As a whole, the app's function will 
not require much security as there will not be much value associated with the data,
so HTTPS connections should be sufficient. 

\section{Software Quality Attributes}
The software is built to be extensible. It can be extended to almost any event at
which there are booths, attractions, or other locations or points of interest. \\ 
Administrators will be able to easily add events to the server, and users will see
and be able to select and pull new events. \\
Event! uses Android Fragments to make the code modular and easily extensible. 
This allows new modules to easily be added for increased functionality. \\
Maintinance should be small, with the only necessary maintinance being in 
database maintinance, and security updates. \\

\chapter{Other Requirements}


\subsection{Naming Conventions}
\begin{itemize}
\item All code is to follow standard Java code naming conventions.
\item Variable names will be in Camel Case, beginning with a lower case letter.\\
\texttt{int variableName;}
\item Method names and Class names are to be camel case, beginning with an upper case letter. \\
\texttt{public class ClassName \{\}}
\item Member level variables are to be in Camel Case, and begin with a lower case character 'm'. \\
\texttt{private int mMemberLevelVariable;}
\end{itemize}

\subsection{Documentation Conventions}
All files are to contain sufficient documentation to be clearly understandable. This 
contains but is not limited to
\begin{itemize}
\item File comments. Each file should contain a comment at the top which contains,
\item File name
\item Author name(s)
\item Date created
\item Date(s) modified
\item brief description of the file contents
\end{itemize}

\texttt{/** \\}
\texttt{* @file FileName.java \\}
\texttt{* @author FirstName LastName\\}
\texttt{* @date 2016 01 15 \\}
\texttt{* @author EditorFitrstName EditorLastName\\}
\texttt{* @date 2016 02 08\\}
\texttt{* This short description is intended to fulfill the requirements of an SRS. \\}
\texttt{*/ \\}

\begin{itemize}
\item Class Comments. All classes should contain a multi-line comment immediately above them which adhires to 
the javadoc standards. They should contain a description of the Class and its intended function. \\
\end{itemize}

\texttt{/** \\}
\texttt{* Class Description used to show the proper documentation \\}
\texttt{* of a class \\}
\texttt{*/ \\}
\texttt{public class ClassName \{\}}

\begin{itemize}
\item Method Comments. All methods should have a multi-line comment immediately with
\item Description
\item Parameter(s)
\item Return
\item Exceptions
\end{itemize}

\texttt{/** \\}
\texttt{* Method description describing the basic features \\}
\texttt{* and function of method\\}
\texttt{* \\}
\texttt{* @param paramName This holds the name of the parameter \\}
\texttt{* @param param2 This is the descriptiion of the second parameter \\}
\texttt{* \\}
\texttt{* @return This is the description of the return value\\}
\texttt{* \\}
\texttt{*/ \\}
\texttt{public class ClassName \{\}}

\end{document}
