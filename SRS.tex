\documentclass{scrreprt}
\usepackage{listings}
\usepackage{underscore}
\usepackage[bookmarks=true]{hyperref}
\usepackage[utf8]{inputenc}
\usepackage[english]{babel}
\hypersetup{
    bookmarks=false,                                % show bookmarks bar?
    pdftitle={Software Requirement Specification},  % title
    pdfauthor={Joshua D. Harshman},                 % author
    pdfsubject={TeX and LaTeX},                     % subject of the document
    pdfkeywords={TeX, LaTeX, graphics, images},     % list of keywords
    colorlinks=true,                                % false: boxed links; true: colored links
    linkcolor=blue,                                 % color of internal links
    citecolor=black,                                % color of links to bibliography
    filecolor=black,                                % color of file links
    urlcolor=purple,                                % color of external links
    linktoc=page                                    % only page is linked
}%
\def\myversion{1.0 }
\date{}
%\title
\usepackage{hyperref}
\begin{document}

\begin{flushright}
    \rule{16cm}{5pt}\vskip1cm
    \begin{bfseries}
        \Huge{SOFTWARE REQUIREMENTS\\ SPECIFICATION}\\
        \vspace{1.9cm}
        for\\
        \vspace{1.9cm}
        Event!\\
        \vspace{1.9cm}
        \LARGE{Version \myversion approved}\\
        \vspace{1.9cm}
        Prepared by Joshua D. Harshman\\Bruce Emehiser\\Aaron Young\\Andrew Macy\\
        \vspace{1.9cm}
        Application Development Team\\
        \vspace{1.9cm}
        \today\\
    \end{bfseries}
\end{flushright}

\tableofcontents


\chapter*{Revision History}

\begin{center}
    \begin{tabular}{|c|c|c|c|}
        \hline
	    Name & Date & Reason For Changes & Version\\
        \hline
	    21 & 22 & 23 & 24\\
        \hline
	    31 & 32 & 33 & 34\\
        \hline
    \end{tabular}
\end{center}

\chapter{Introduction}

\section{Purpose}
$<$Identify the product whose software requirements are specified in this 
document, including the revision or release number. Describe the scope of the 
product that is covered by this SRS, particularly if this SRS describes only 
part of the system or a single subsystem.$>$

\section{Document Conventions}
$<$Describe any standards or typographical conventions that were followed when 
writing this SRS, such as fonts or highlighting that have special significance.  
For example, state whether priorities  for higher-level requirements are assumed 
to be inherited by detailed requirements, or whether every requirement statement 
is to have its own priority.$>$

\section{Intended Audience and Reading Suggestions}
$<$Describe the different types of reader that the document is intended for, 
such as developers, project managers, marketing staff, users, testers, and 
documentation writers. Describe what the rest of this SRS contains and how it is 
organized. Suggest a sequence for reading the document, beginning with the 
overview sections and proceeding through the sections that are most pertinent to 
each reader type.$>$

\section{Project Scope}
$<$Provide a short description of the software being specified and its purpose, 
including relevant benefits, objectives, and goals. Relate the software to 
corporate goals or business strategies. If a separate vision and scope document 
is available, refer to it rather than duplicating its contents here.$>$

\section{References}
$<$List any other documents or Web addresses to which this SRS refers. These may 
include user interface style guides, contracts, standards, system requirements 
specifications, use case documents, or a vision and scope document. Provide 
enough information so that the reader could access a copy of each reference, 
including title, author, version number, date, and source or location.$>$


\chapter{Overall Description}

\section{Product Perspective}
The Event app is a stand alone self contained product
$<$Describe the context and origin of the product being specified in this SRS.  
For example, state whether this product is a follow-on member of a product 
family, a replacement for certain existing systems, or a new, self-contained 
product. If the SRS defines a component of a larger system, relate the 
requirements of the larger system to the functionality of this software and 
identify interfaces between the two. A simple diagram that shows the major 
components of the overall system, subsystem interconnections, and external 
interfaces can be helpful.$>$

\section{Product Functions}
Event! is intended to allow users to find interesting locations, to influence them 
to visit those locations, to share the experience with friends, and collect badges 
which they can redeem for some nominal reward. Because of this, we have the 
following functions: 
\begin{itemize}
\item User authentication
\item GPS and QR location check in
\item Social media hooks for sharing events/waypoints
\end{itemize}
With these features, users will be allowed to log in to the app, view events, 
view waypoints, share events and waypoints via social media, and collect 
badges at waypoints via GPS and QR.

\section{User Classes and Characteristics}
Event! appeals to all age ranges, with emphasis on the young who will be visiting 
events, art walks, and the like.

\section{Operating Environment}
Event! will run on Android API 16 and newer android devices. It is designed for 
mobile phones, but it can also be run on Android tablets, although the screen 
sizes are not spesifically supported. The phones/tablets must have a rear facing 
camera to be able to utilize the QR code check in feature.

\section{Design and Implementation Constraints}

$<$Describe any items or issues that will limit the options available to the 
developers. These might include: corporate or regulatory policies; hardware 
limitations (timing requirements, memory requirements); interfaces to other 
applications; specific technologies, tools, and databases to be used; parallel 
operations; language requirements; communications protocols; security 
considerations; design conventions or programming standards (for example, if the 
customer’s organization will be responsible for maintaining the delivered 
software).$>$

\section{User Documentation}
$<$List the user documentation components (such as user manuals, on-line help, 
and tutorials) that will be delivered along with the software. Identify any 
known user documentation delivery formats or standards.$>$

\section{Assumptions and Dependencies}
The app is dependent on a database back end for pulling events and updating 
content. Without the back end it will be incapable of pulling event data and 
providing the user with information.


\chapter{External Interface Requirements}

\section{User Interfaces}

The user interface will be built with Android XML layouts. It will feature
\begin{itemize}
\item Paging introduction view
\item Navigation drawer
\item Login screen
\item Event view screen
\item Waypoint view screen
\item QR code scanner screen
\end{itemize}

\section{Hardware Interfaces}
Each device must have a rear-facing camera capable of producing JPEG photos, 
and auto focus. This will be used by the QR code scanner for event waypoint checkin.
Event! will also use network to send and recieve information from a server.

\section{Software Interfaces}
Event! will use ZXing QR code scanner API for decoding and creating QR codes for 
event and waypoint checkin. \\
Event! will use Apache server with PHP and MySQL database for maintaining event 
information and user information.

\section{Communications Interfaces}
Event! will communicate with the server via HTTPS connections, for getting event 
data, and storing user data. \\
Event will also communicate with social media outlets Google (for user authentication), 
and Google+, Facebook, Twitter, and Instagram for social media sharing via the 
APIs provided by the companies.

\chapter{System Features}
$<$This template illustrates organizing the functional requirements for the 
product by system features, the major services provided by the product. You may 
prefer to organize this section by use case, mode of operation, user class, 
object class, functional hierarchy, or combinations of these, whatever makes the 
most logical sense for your product.$>$

\section{System Feature 1}
$<$Don’t really say “System Feature 1.” State the feature name in just a few 
words.$>$

\subsection{Description and Priority}
$<$Provide a short description of the feature and indicate whether it is of 
High, Medium, or Low priority. You could also include specific priority 
component ratings, such as benefit, penalty, cost, and risk (each rated on a 
relative scale from a low of 1 to a high of 9).$>$

\subsection{Stimulus/Response Sequences}
$<$List the sequences of user actions and system responses that stimulate the 
behavior defined for this feature. These will correspond to the dialog elements 
associated with use cases.$>$

\subsection{Functional Requirements}
$<$Itemize the detailed functional requirements associated with this feature.  
These are the software capabilities that must be present in order for the user 
to carry out the services provided by the feature, or to execute the use case.  
Include how the product should respond to anticipated error conditions or 
invalid inputs. Requirements should be concise, complete, unambiguous, 
verifiable, and necessary. Use “TBD” as a placeholder to indicate when necessary 
information is not yet available.$>$

$<$Each requirement should be uniquely identified with a sequence number or a 
meaningful tag of some kind.$>$

REQ-1:	REQ-2:

\section{System Feature 2 (and so on)}


\chapter{Other Nonfunctional Requirements}

\section{Performance Requirements}
$<$If there are performance requirements for the product under various 
circumstances, state them here and explain their rationale, to help the 
developers understand the intent and make suitable design choices. Specify the 
timing relationships for real time systems. Make such requirements as specific 
as possible. You may need to state performance requirements for individual 
functional requirements or features.$>$

\section{Safety Requirements}
$<$Specify those requirements that are concerned with possible loss, damage, or 
harm that could result from the use of the product. Define any safeguards or 
actions that must be taken, as well as actions that must be prevented. Refer to 
any external policies or regulations that state safety issues that affect the 
product’s design or use. Define any safety certifications that must be 
satisfied.$>$

\section{Security Requirements}
$<$Specify any requirements regarding security or privacy issues surrounding use 
of the product or protection of the data used or created by the product. Define 
any user identity authentication requirements. Refer to any external policies or 
regulations containing security issues that affect the product. Define any 
security or privacy certifications that must be satisfied.$>$

\section{Software Quality Attributes}
$<$Specify any additional quality characteristics for the product that will be 
important to either the customers or the developers. Some to consider are: 
adaptability, availability, correctness, flexibility, interoperability, 
maintainability, portability, reliability, reusability, robustness, testability, 
and usability. Write these to be specific, quantitative, and verifiable when 
possible. At the least, clarify the relative preferences for various attributes, 
such as ease of use over ease of learning.$>$

\section{Business Rules}
$<$List any operating principles about the product, such as which individuals or 
roles can perform which functions under specific circumstances. These are not 
functional requirements in themselves, but they may imply certain functional 
requirements to enforce the rules.$>$


\chapter{Other Requirements}
$<$Define any other requirements not covered elsewhere in the SRS. This might 
include database requirements, internationalization requirements, legal 
requirements, reuse objectives for the project, and so on. Add any new sections 
that are pertinent to the project.$>$

\subsection{Naming Conventions}
\begin{itemize}
\item All code is to follow standard Java code naming conventions.
\item Variable names will be in Camel Case, beginning with a lower case letter.\\
\texttt{int variableName;}
\item Method names and Class names are to be camel case, beginning with an upper case letter. \\
\texttt{public class ClassName \{\}}
\item Member level variables are to be in Camel Case, and begin with a lower case character 'm'. \\
\texttt{private int mMemberLevelVariable;}
\end{itemize}

\subsection{Documentation Conventions}
All files are to contain sufficient documentation to be clearly understandable. This 
contains but is not limited to
\begin{itemize}
\item File comments. Each file should contain a comment at the top which contains,
\item File name
\item Author name(s)
\item Date created
\item Date(s) modified
\item brief description of the file contents
\end{itemize}

\texttt{/** \\}
\texttt{* @file FileName.java \\}
\texttt{* @author FirstName LastName\\}
\texttt{* @date 2016 01 15 \\}
\texttt{* @date 2016 02 08\\}
\texttt{* This short description is intended to fulfill the requirements of an SRS. \\}
\texttt{*/ \\}

\begin{itemize}
\item Class Comments. All classes should contain a multi-line comment immediately above them which adhires to 
the javadoc standards. They should contain a description of the Class and its intended function. \\
\end{itemize}

\texttt{/** \\}
\texttt{* Class Description used to show the proper documentation \\}
\texttt{* of a class \\}
\texttt{*/}
\texttt{public class ClassName \{\}}

\begin{itemize}
\item Method Comments. All methods should have a multi-line comment immediately with
\item Description
\item Parameter(s)
\item Return
\item Exceptions
\end{itemize}

\texttt{/** \\}
\texttt{* Method description describing the basic features \\}
\texttt{* and function of method\\}
\texttt{* \\}
\texttt{* @param paramName This parameter is useful because it shows param notation in comments \\}
\texttt{* @param param2 This shows the second parameter \\}
\texttt{* \\}
\texttt{* @return This is the description of the return value\\}
\texttt{* \\}
\texttt{*/}
\texttt{public class ClassName \{\}}


\section{Appendix A: Glossary}
%see https://en.wikibooks.org/wiki/LaTeX/Glossary
$<$Define all the terms necessary to properly interpret the SRS, including 
acronyms and abbreviations. You may wish to build a separate glossary that spans 
multiple projects or the entire organization, and just include terms specific to 
a single project in each SRS.$>$

\section{Appendix B: Analysis Models}
$<$Optionally, include any pertinent analysis models, such as data flow 
diagrams, class diagrams, state-transition diagrams, or entity-relationship 
diagrams.$>$

\section{Appendix C: To Be Determined List}
$<$Collect a numbered list of the TBD (to be determined) references that remain 
in the SRS so they can be tracked to closure.$>$

\end{document}
